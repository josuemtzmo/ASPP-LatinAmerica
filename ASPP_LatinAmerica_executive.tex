\documentclass{article}[11pt]
\usepackage[utf8]{inputenc}
\usepackage[affil-it]{authblk}
\usepackage{tabularx,colortbl}
\usepackage{hyperref}
\usepackage[dvipsnames]{xcolor}
\usepackage[margin=1.4in]{geometry}

\usepackage{trackchanges}

%%   TrackChanges adds the following 5 commands to allow for collaborative
%%   editing of LaTeX documents.
%%
%%     \add[editor]{text to add}
%%     \remove[editor]{text to remove}
%%     \change[editor]{text to remove}{text to add}
%%     \annote[editor]{text to annotate}{note}
%%     \note[editor]{note}

\title{%
\Large Advanced Scientific Programming in Python  -- Latin America
\\ \vspace{0.2cm}
\large Summer School Proposal  Executive Summary  [Draft \today]}
%\\ \vspace{0.2cm}
%\large
\date{}
\author{}

\begin{document}
\maketitle


\vspace*{-1.2cm}
\section*{Introduction}
Scientific software development is no longer reserved for a reduced number of
scientist working in highly specialized branches in the field of computer
science. Powerful programming techniques are now available to the general
scientific community and across many academic and non-academic institutions and
fields, making \emph{the ability to program} a highly desired and important
skill to conduct state of the art research. However, there is a considerable
lack of education regarding such programming practices and techniques in
the community of scientists who are generally not properly trained for the
creation and maintenance of scientific software. Very often scientists spend
(waste) a significant amount time writing, maintaining and debugging
sub-optimal self-written code, often reinventing the wheel, instead of being
able to focus their efforts on their topics of research. The result is the
creation of poorly written scientific code which is not efficient and becomes
very limited and often obsolete very fast.

With the explicit intent of responding to these needs in the scientific
community the \emph{Advanced Scientific Programming in Python Summer School}
was launched in Europe twelve years ago, providing the scientific community
with a unique opportunity to learn techniques and tools to develop efficient
and sustainable code with hands-on experience on the subset of these techniques
that matter most, teaching them clean language design, extensibility of code,
and good practices in scientific computing and data visualization. Each year,
this self funded effort has attracted scientist who want to acquire software
engineering skills to tackle new challenges in their areas of expertise for
the  benefit of the scientific community. The summer school grew quickly, and
in recent years, the number of applications  to the school has grown to be over
300 applications per year, for the 30 places typically available (with a 50/50
gender split). Its success and worldwide demand has lead to the creation of the \emph{Advanced Scientific Programming in Python Summer School - Asia - Pacific}, and for now two years the school has been offered to a public of the Asian continent.

The purpose of this project is to bring the teachings and benefits of the
school to a Latin American audience and its scientific community. The number of
latin american scientist who are directly working with scientific software
development has dramatically grown in the past few years without any effort of
alleviating such problems. For this, we propose the creation of the
\emph{Advanced Scientific Programming in Python Summer School - Latin America}
with the aim of bringing together scientists from all latin american countries
in order to participate in the first ASPP in the American continent. During the
7-day intensive course, participants will explore and learn the fundamentals
ideas of scientific code development from leading experts in the field.After
the summer school the participants will be able to return to their home
countries and benefit to the development of the scientific community of each
nation.  We aim to hold the first instance of the school during the month of
July of 2020 in Mexico City due to its position as a leading developing country
of latin america and due to its being home of the largest and most prestigious
research institution in the Ibero-American world.

In line with the \emph{open source} commitment of the scientific Python
community, the complete material, lectures, exercises, solutions, and code
produced will be available online under a Creative Commons license. The summer
school will not charge fees, and we expect it to remain free from commercial
interests. This opens up the possibility of using the material for the
development and creation of new courses tailored to the specific needs of a
specific target audience. Many social programs can be developed in order to
attack the problem from an earlier stage, creating courses for various stages
of education for example, high school students, college students, science
teachers, etc.

The previous success of the other instances of the summer school is very highly
related to the engaging and interactive environment that has been created by a
group of highly capable and accessible scientists who have created a teaching
methodology which fosters learning. The teaching and learning approach itself
is largely inspired by the Agile technique from software engineering, which
promotes flexible and responsive experience, and nurtures learning
opportunities. In order to reproduce this environment, we aim to hold the
school in a remote location in Mexico City in order to create a profound
learning experience away from distractions and daily routines. Informal chats
during shared meals and recreation time and during coding exercises will
provide opportunities to all participants to form close professional and
personal relationships, fostering future collaborations and openness within the
next generation of Latin-American scientists.


\section*{Structure of the Summer School}
The summer school Advanced Scientific Programming in Python - Latin America
(ASPP~-~LA) will follow the structure and curriculum of the previously
organized schools in Europe (2008-2019) and the Asia-Pacific region (2018 and
2019).

During the summer school, we will teach a selection of advanced programming
techniques and best practices in software development (which are standard in
the industry) to researchers in the natural and social sciences, engineering
and mathematics. The school will  consist of lectures and discussions along
with practical hands-on exercises which will supplement the lectures and will
let participants practice the techniques. All of the newly acquired skills will
be tested in a real programming project: participants will team up to develop
an entertaining scientific computer game.

We use the Python programming language for the entire course. Python works as a
simple programming language for beginners, but more importantly, it also works
great in scientific simulations and data analysis. These features make the
language a very popular choice in almost any branch of science. We show how
clean language design, ease of extensibility, and the great wealth of open
source libraries for scientific computing and data visualization are driving
Python to become a standard tool for the programming scientist.

\section*{Target Audience}
The summer school is targeted at master or phd students, as well as post-docs
and research scientists from Latin American countries. Competence in Python or
in another language such as Java, C/C++, MATLAB, or Mathematica is required,
and basic knowledge of Python is assumed. The course is planned to allow for an
attendance of 30 international students (50/50 gender split), who fund their
own travel and stay for a week. Importantly, participation in the school is
free, i.e. no fee is charged, however participates should take care of travel,
living, and accommodation expenses by themselves. We aim to provide travel
allowance for a number of participants if resources are sufficient (see section
``Funding" below).

\section*{Preliminary Program}
These lectures will build from educational materials given during the previous
ASPP-Europe and ASPP-Asia-Pacific schools. The contents of the lectures, the
exercises and the programming project are developed by all faculty members
together in order to ensure a coherent set of materials, which gets further
refined by integrating the student feedback collected by a survey at the end of
each instance of the school.

\begin{itemize}
\item Day 1: Best Programming Practices - Best Practices for Scientific
Computing - Version control with Git and how to contribute to open source with
GitHub - Object Oriented Programming and Design Patterns.

\item Day 2: Software Carpentry - Test-driven development, unit testing and
quality assurance - Debugging, profiling and benchmarking techniques -
Advanced Python 1: idioms, useful built-in data structures, generators.

\item Day 3: Scientific Tools for Python – Advanced NumPy – The Quest for
Speed 1: Interfacing to C with Cython – Programming in teams.

\item Day 4: The Quest for Speed 2: Writing parallel applications in Python  –
Programming project.

\item Day 5: Efficient Memory Management – Advanced Python 2: decorators and
context managers – Programming project.

\item Day 6: Practical Software Development – Programming project – The
Scientific Game Tournament.
\end{itemize}

% \begin{itemize}
%     \item Best Programming Practices for Scientific Computing - Version
% control with Git and how to contribute to open source with GitHub - Object
% Oriented Programming and Design Patterns.
%     \item Tidy data analysis and visualization
%     \item Testing and debugging scientific code
%     \item Advanced NumPy
%     \item Organizing, documenting, distributing  scientific code and
% continuous integration
%     \item Advanced scientific Python: context managers and generators
%     \item Writing parallel applications in Python
%     \item Profiling and speeding up scientific code with Cython and Numba
% \end{itemize}

\section*{Preliminary Faculty}
The number of speakers is approximately 10, and they also act as tutors for
the students during the course. All speakers are computational scientists or
software developers with a PhD in relevant disciplines. Speakers do not
receive any honorarium, and take part in the summer school on their spare time:
\begin{itemize}
\item Tiziano Zito, Institute of Psychology, Humboldt Universit\"at zu Berlin,
Germany.

\item Rike-Benjamin Schuppner, Institute for Theoretical Biology, Humboldt-
Universit\"at zu Berlin, Germany.

\item St\'efan van der Walt, Berkeley Institute for Data Science, University
of California at Berkeley, USA.

\item Madicken Munk, Postdoctoral Researcher at the  Data Exploration Lab,
National Center for Supercomputing Applications, University of Illinois at
Urbana-Champaign, USA.

\item Kathryn D. Huff, Assistant Professor, Department of Nuclear, Plasma, and
Radiological Engineering, University of Illinois at Urbana-Champaign, USA.

\item Ricardo M\'endez Fragoso, Professor of Physics, Faculty of Sciences of
the National Autonomous University of Mexico.
\end{itemize}


\section*{Organizing Committee}
\begin{itemize}
\item Carlos Echeverr\'ia Serur, PhD Candidate at the Institut f\"ur
Mathematik of the Technische Universit\"at Berlin (Alumnus ASPP-Europe
2018).

\item Luis Garc\'ia Ramos, PhD Candidate at the Institut f\"ur Mathematik
of the Technische Universit\"at Berlin (Alumnus ASPP-Europe 2018).

\item Josu\'e Mart\'inez Moreno, PhD Candidate at Research School of Earth
Sciences of the Australian National University, Canberra (Alumnus ASPP-
APAC 2019).

\item Ricardo M\'endez Fragoso, Professor at Department of Physics, School
of Sciences of the National Autonomous University of Mexico, (Invited to
ASPP-Europe 2019).

\item Florencia Noriega, Software Engineering Lecturer at CODE University
of Applied Sciences, Berlin (Alumna ASPP-Europe 2016).

\item Tiziano Zito, from the Institut f\"ur Psychologie of the Humboldt
Universit\"at zu Berlin (Co-Founder of ASPP-Europe).
\end{itemize}

\section*{Funding}
The total budget of the event varies in range between 10,000 and 20,000 USD,
depending on how much we are able to obtain as donations from sponsors and/or
partner organizations (facilities, lecture rooms, computer laptops,
accommodation for faculty and students).

The budget will be used to fund travel, accommodation and daily expenses of
faculty members and organizers. It will also cover the rent of 15 laptops
(1,500 USD), organize catering for breaks (400 USD) and lunch for the speakers
(700 USD), as well as 2 social events (2,000 USD). Students pay their own
travel and accommodation costs. In order to open the school to a population of
students from developing Latin American countries who do not have access to
funding, we would like to offer financial support to fund travel or
accommodation expenses for a number of international and local students (12,000
USD)

% \subsection*{Estimated costs per Faculty Member (should be covered by the
% organizers)}
% \begin{itemize}
%     \item Flight (from Europe or US): Approx. 800USD
%     \item Accommodation in CDMX: Approx. 400USD
%     \item Local Transport: Approx. 100USD
%     \item Food for the entire duration of the conference: Approx. 200USD
% \end{itemize}
%
% \subsection*{Estimated costs of Organization (should be covered by the
% organizers)}
% \begin{itemize}
%   \item Computer Rental (40 computers): Approx. ???USD
%   \item Venue Rental: Approx. ???USD
%   \item Social Activity 1 (Trip to Teotihuac\'an): Approx. 1000USD
%   \item Social Activity 2 (Trip to ???): Approx. 1000USD
%   \item Airport-Venue Transport: Approx. ???USD
% \end{itemize}
%
% \subsection*{Estimated average costs per Participant (should be covered by
% the participants)}
% \begin{itemize}
%   \item Flight (From a Latin American Country): Approx. 500USD
%   \item Accommodation in CDMX: Approx. 300USD
%   \item Local Transport: Approx. 50USD
%   \item Food for the entire duration of the conference: Approx. 150USD
% \end{itemize}

\subsection*{Tentative Partner Organizations (Sponsors/Co-Organizers)}
\begin{itemize}
    \item Secretar\'ia de Ciencia, Tecnolog\'ia e Innovaci\'on de la CDMX
    (SECTEI).
    \item Facultad de Ciencias, UNAM (FCIENCIAS-UNAM).
    \item Sociedad Mexiacan de F\'isica (SMF)
    \item Instituto de Investigaciones en Matem\'aticas Aplicadas (IIMAS-UNAM)
    \item Consejo Nacional de Ciencia y Tecnolog\'ia (CONACYT)
    \item Instituto Latinoamericano de la Comunicaci\'on Educativa (ILCE)
    \item Instituto Politecnico Nacional (IPN)
\end{itemize}

\section*{Additional Information}
More information and teaching materials from previous instances of the school
can be found at the following website: \url{http://python.g-node.org}

\end{document}
