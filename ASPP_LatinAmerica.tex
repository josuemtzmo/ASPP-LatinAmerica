\documentclass{article}
\usepackage[utf8]{inputenc}
\usepackage[affil-it]{authblk}
\usepackage{tabularx,colortbl}

\usepackage[dvipsnames]{xcolor}

\usepackage{trackchanges}

%%   TrackChanges adds the following 5 commands to allow for collaborative 
%%   editing of LaTeX documents.
%%
%%     \add[editor]{text to add}
%%     \remove[editor]{text to remove}
%%     \change[editor]{text to remove}{text to add}
%%     \annote[editor]{text to annotate}{note}
%%     \note[editor]{note}

\title{Advanced Scientific Programming in Python - Latin America}
\date{July 2020}
\author{Location}

\begin{document}
\maketitle

\begin{center}
\end{center}

\section*{Organization Committee}
\begin{itemize}
    \item Josu\'e Mart\'inez Moreno (head) from the Research School of Earth Sciences of the Australian National University
    \item Richardo M\'endez Fragoso form the School off Sciences of the National Autonomous University of Mexico.
    \item Carlos Echeverr\'ia Serur from the Institut für Mathematik of the Technische Universität Berlin.
    \item Luis Garc\'ia Ramos from the Institut für Mathematik of the Technische Universität Berlin.
    \item \note[add]{Look for and add someone from currently in Latin America}
    \item \note[add]{We should add someone else in Mexico so they can help Rich later on with the local organization (@Rich Could you suggest someone?).}
    %\item Tiziano Zito from the Institut für Mathematik of the Technische Universität Berlin.
\end{itemize}

\section*{Faculty}

\begin{itemize}
    \item Josu\'e Mart\'inez Moreno
    \item Ricardo M\'endez Fragoso
    \item Carlos Echeverr\'ia Serur
    \item Luis Garc\'ia Ramos 
    \item Tiziano Zito
    \item \add[Josue]{Who else may be interested? Ideally we should invite people from the US, and America Latina}
\end{itemize}

Note: The budget will fund travel, accommodation and \textit{Per Diem} expenses of lecturers and organizers. Typically, organizers also serve as lecturers. The budget may not allow for funding of organizers who do not actively participate in the school as lecturers. Additional budget will be use to support accommodation or travel expenses of international and local students.

\section*{Scope}

The Advanced Scientific Programming in Python (ASPP) - Latin America will follow the structure of the previous organized schools in Europe and Asia-Pacific. This intensive programming course aims to teach the rudiments of scientific python and how to use it in a reproducible, efficient and academic way. We are concerned with the current use of programming languages without fully understanding the foundations and their potential. We are also interested in concepts and tools to use programming as an open-source community for the purpose of scientific development. With only a week of lectures, we accept that these broad aims will be at best be partially realized. Nonetheless, the students will be exposed to a suite of resources which, if studied and practiced, will help to train a future scientists able to improve their codes, but also to make it reusable and easy to understand in case someone wants to reproduce their results.

The course will be taught in a traditional manner with spoken lectures and discussions, along with practical sections which will supplement lectures and will encourage students to engage and take advantage of the lectures. The student is assumed to have prior knowledge on any coding language and a basic understanding of Python. The lectures will move at a rapid, though pedagogical pace. In general, the material covered will be more than the average amount an student could digest in a single week. Therefore to take fully advantage of this course, we encourage students to read the pre-course and post-course study. This particular topic requires devoted engagement and interaction not only during the lectures but after the course is finished. 

This course is not \textit{per se} a how to code course, though some basics will be cover. This is also not an programming engineering or computer science course. Rather, the lectures are focused on how scientists can optimize their work-flow, as well as to facilitate the reproducibility, collaboration and maintenance of their codes. Correspondingly, there will be a strong hands-on component, where students are encouraged and expected to participate, discuss and proactively learn not only from the faculty but also from other students. 

\section*{Participants}

Participation is free, i.e. no fee is charged, however participates should take care of travel, living, and accommodation expenses by themselves.

\subsection{Prerequisites}
There are only two requirements to ASPP - Latin America:
\begin{itemize}
    \item Applicants are recommended to know programming and the basics of Python to get the most out of this Summer School.
    \item Applicants should provide proof of their English proficiency, as all lectures will be on presented in English. At least, applicants should be able to understand lectures in English. Applicants are not required to provide results from a proficiency test. \note[Josue]{Is there something else to add? Shall we ask for a reference letter or something?}.
\end{itemize}

\section*{Lecture Topics}

These lectures will build from feedback obtained during the previous ASPP-Europe and ASPP-Asiapacific schools (topics include some of the potential speakers).

\begin{itemize}
    \item Version control (GitHub) - 
    \item Tidy data analysis and visualization - Rich \note[Josue]{Perhaps change the structure and remove Tidy data?s}
    \item Testing and debugging scientific code - Luis
    \item Advanced NumPy - Carlos
    \item Organizing, documenting, distributing  scientific code and contimuous integration - 
    \item Advanced scientific Python: context managers and generators - 
    \item Writing parallel applications in Python - 
    \item Profiling and speeding up scientific code with Cython and numba - 
\end{itemize}

\subsection*{Schedule}

\begin{center}
\begin{tabularx}{\textwidth}{|X|X|X|X|}
\hline
\rowcolor{Aquamarine}
\multicolumn{4}{|c|}{Monday XX July 2020}\\
\hline
\rowcolor[gray]{.7}
Time & Topic & Speaker & Tutors \\
\hline
9:00-9:30 & Introduction &  &   \\
\hline
9:30-11:00 & Git \& Github &  &  \\
\hline
\rowcolor[gray]{.9}
11:00-11:30 & \multicolumn{3}{c|}{Coffee Break} \\
\hline
11:30-13:00 & Git \& Github &  &  \\
\hline
\rowcolor[gray]{.9}
13:00-14:00 & \multicolumn{3}{c|}{Lunch Break} \\
\hline
14:00-15:00 & Git \& Github &  &  \\
\hline
15:00-16:00 & Data visualization & Ricardo M\'endez &  \\
\hline
\rowcolor[gray]{.9}
16:00-16:30 & \multicolumn{3}{c|}{Coffee Break} \\
\hline
16:30-18:00 & Data visualization & Ricardo M\'endez &  \\
\hline
\end{tabularx}
\end{center}

\begin{center}
\begin{tabularx}{\textwidth}{|X|X|X|X|}
\hline
\rowcolor{Aquamarine}
\multicolumn{4}{|c|}{Tuesday XX July 2020}\\
\hline
\rowcolor[gray]{.7}
Time & Topic & Speaker & Tutors \\
\hline
9:00-9:30 & Testing \& debugging &  &   \\
\hline
9:30-11:00 & Testing \& debugging &  &  \\
\hline
\rowcolor[gray]{.9}
11:00-11:30 & \multicolumn{3}{c|}{Coffee Break} \\
\hline
11:30-13:00 & Testing \& debugging &  &  \\
\hline
\rowcolor[gray]{.9}
13:00-14:00 & \multicolumn{3}{c|}{Lunch Break} \\
\hline
14:00-15:00 & Advanced Numpy &  &  \\
\hline
15:00-16:00 & Advanced Numpy  &  &  \\
\hline
\rowcolor[gray]{.9}
16:00-16:30 & \multicolumn{3}{c|}{Coffee Break} \\
\hline
16:30-18:00 &  &  &  \\
\hline
\end{tabularx}
\end{center}
\note[Josue]{I would like to add some Scipy, what do you think?}

\begin{center}
\begin{tabularx}{\textwidth}{|X|X|X|X|}
\hline
\rowcolor{Aquamarine}
\multicolumn{4}{|c|}{Wendsday XX July 2020}\\
\hline
\rowcolor[gray]{.7}
Time & Topic & Speaker & Tutors \\
\hline
9:00-11:00 & Organizing, documenting, distributing \& CI &  &  \\
\hline
\rowcolor[gray]{.9}
11:00-11:30 & \multicolumn{3}{c|}{Coffee Break} \\
\hline
11:30-13:00 &  &  &  \\
\hline
\rowcolor[gray]{.9}
13:00-XX:XX & \multicolumn{3}{c|}{Social activity} \\
\hline
\end{tabularx}
\end{center}
\note[Josue]{Do we want a social activity?}

\begin{center}
\begin{tabularx}{\textwidth}{|X|X|X|X|}
\hline
\rowcolor{Aquamarine}
\multicolumn{4}{|c|}{Thursday XX July 2020}\\
\hline
\rowcolor[gray]{.7}
Time & Topic & Speaker & Tutors \\
\hline
9:00-10:30 & Parallel Python &  &   \\
\hline
\rowcolor[gray]{.9}
10:30-11:00 & \multicolumn{3}{c|}{Coffee Break} \\
\hline
11:30-13:00 & Parallel Python - Pycuda &  &  \\
\hline
\rowcolor[gray]{.9}
13:00-14:00 & \multicolumn{3}{c|}{Lunch Break} \\
\hline
14:00-14:30 & Project Introduction &  &  \\
\hline
14:30-16:00 & Programming Project &  &  \\
\hline
\rowcolor[gray]{.9}
16:00-16:30 & \multicolumn{3}{c|}{Coffee Break} \\
\hline
16:30-18:00 & Programming Project &  &  \\
\hline
\end{tabularx}
\end{center}

\begin{center}
\begin{tabularx}{\textwidth}{|X|X|X|X|}
\hline
\rowcolor{Aquamarine}
\multicolumn{4}{|c|}{Friday XX July 2020}\\
\hline
\rowcolor[gray]{.7}
Time & Topic & Speaker & Tutors \\
\hline
9:00-10:30 & Profiling, Cython \& numba &  &   \\
\hline
\rowcolor[gray]{.9}
10:30-11:00 & \multicolumn{3}{c|}{Coffee Break} \\
\hline
11:30-13:00 & Profiling, Cython \& numba  &  &  \\
\hline
\rowcolor[gray]{.9}
13:00-14:00 & \multicolumn{3}{c|}{Lunch Break} \\
\hline
14:00-16:00 & Programming Project &  &  \\
\hline
\rowcolor[gray]{.9}
16:00-16:30 & \multicolumn{3}{c|}{Coffee Break} \\
\hline
16:30-18:00 & Programming Project &  &  \\
\hline
\end{tabularx}
\end{center}

\begin{center}
\begin{tabularx}{\textwidth}{|X|X|X|X|}
\hline
\rowcolor{Aquamarine}
\multicolumn{4}{|c|}{Saturday XX July 2020}\\
\hline
\rowcolor[gray]{.7}
Time & Topic & Speaker & Tutors \\
\hline
9:00-10:30 & Programming Project &  &   \\
\hline
\rowcolor[gray]{.9}
10:30-11:00 & \multicolumn{3}{c|}{Coffee Break} \\
\hline
11:30-13:00 & Programming Project &  &  \\
\hline
\rowcolor[gray]{.9}
13:00-14:00 & \multicolumn{3}{c|}{Lunch Break} \\
\hline
14:00-16:00 & Programming Project &  &  \\
\hline
\rowcolor[gray]{.9}
16:00 & \multicolumn{3}{c|}{Repository Freeze} \\
\rowcolor[gray]{.9}
16:00-16:30 & \multicolumn{3}{c|}{Coffee Break} \\
\hline
16:30-16:45 & How to contribute to ASPP &  &  \\
\hline
\rowcolor[gray]{.9}
16:45-17:00 & \multicolumn{3}{c|}{Programming project - Learning report}\\
\hline
\rowcolor[gray]{.9}
17:00-18:00 & \multicolumn{3}{c|}{Friendly Tournament} \\
\hline
\rowcolor[gray]{.9}
18:00-XX:XX & \multicolumn{3}{c|}{Final Social Event} \\
\hline
\end{tabularx}
\end{center}

\note[Josue]{I've remove 1 hour from the lunch break so people can go earlier home, do you think we should use that hour or should'nt?}

\subsection*{Additional materials}

\subsection*{Partner Organizations (Funding)}
\begin{itemize}
    \item School of Science (UNAM) - 
    \item CONACYT - 
    
    \item \add[Josue]{More Options}
\end{itemize}

\section*{Policies}

\subsection*{Inclusiveness}

\subsection*{Behavior}

\subsection*{}


\end{document}
