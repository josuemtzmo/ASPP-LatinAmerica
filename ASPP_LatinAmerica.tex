\documentclass{article}[11pt]
\usepackage[utf8]{inputenc}
\usepackage[affil-it]{authblk}
\usepackage{tabularx,colortbl}
\usepackage{hyperref}
\usepackage[dvipsnames]{xcolor}

\usepackage{trackchanges}

%%   TrackChanges adds the following 5 commands to allow for collaborative
%%   editing of LaTeX documents.
%%
%%     \add[editor]{text to add}
%%     \remove[editor]{text to remove}
%%     \change[editor]{text to remove}{text to add}
%%     \annote[editor]{text to annotate}{note}
%%     \note[editor]{note}

\title{%
Summer School: Advanced Scientific Programming in Python - Latin America 
\\ \vspace{0.2cm}
\large Project Proposal [Draft \today]}
\date{July 2020}
\author{M\'exico City, M\'exico}

\begin{document}
\maketitle

\begin{center}
\end{center}

\section*{Scope}
The summer school Advanced Scientific Programming in Python (ASPP) - Latin 
America will follow the structure and curriculum of the previously organized 
schools in Europe (2008-2019) and the Asia-Pacific region (2018, 2019). 
This highly successful summer schools have shown that there is a need for 
teaching scientists the principles of \emph{scientific computing and software 
engineering} in the Python programming language. The goal of this intensive 
programming workshop is to teach a selection of advanced programming techniques 
and best practices in  software development which are standard in the industry, 
to researchers in the natural and social sciences, engineering and mathematics.\\

The tools and concepts introduced in this workshop will help researchers spend 
less time writing, debugging and understanding computer programs. In addition, 
we expect that after attending this course participants will be able to 
contribute to the open-source scientific software community. With only a 
week of lectures, we accept that these ambitious goals will be at best 
partially realized. Nonetheless, the participants will be exposed to a suite 
of resources which, if studied and practiced, will help to train a generation 
of future scientists to be able to write better code, but also to make it 
reusable and easy to understand in case someone wants to reproduce their results.\\

The course will  consist of lectures and discussions along with practical 
hands-on exercises which will supplement the lectures and will let participants 
practice the techniques. The participants are assumed to have prior knowledge 
(at the level of a competent user) on some high-level programming language used 
in scientific computing (for example, C, C++, Java, Fortran, MATLAB, etc.) and 
a basic knowledge of Python. The lectures will move at a rapid, though 
pedagogical pace. In general, the material covered will be more than the 
average amount that a participant can digest in a single week. Therefore, 
it will be mandatory that the participants familiarize themselves with some 
(online) prerequisite  material that we will provide. Moreover, we encourage 
participants to continue with further study and implementation of the 
techniques after the course.\\

The course will not be an introductory course to the Python programming 
language, though questions on basic features of the language will be refreshed 
as we go along the lectures if necessary. It will also not be a specialized 
course on software engineering or computer science. Rather, the lectures will 
be focused on the needs of scientists who need to optimize their software 
work-flow. Correspondingly, there will be a strong hands-on component, where 
participants are encouraged and expected to participate, discuss and proactively 
learn not only from the faculty but also from other participants. 

%(try to address the needs of all scientist from various disciplines).
%(as well as to facilitate the reproducibility, collaboration and maintenance of their codes)

\section*{Organizing Committee}
\begin{itemize}
    \item Carlos Echeverr\'ia Serur, PhD Student at the Institut f\"ur Mathematik of the Technische Universit\"at Berlin (Alumnus ASPP-Europe 2018).

    \item Luis Garc\'ia Ramos, PhD Student at the Institut f\"ur Mathematik of the Technische Universit\"at Berlin (Alumnus ASPP-Europe 2018).
    
    \item Josu\'e Mart\'inez Moreno (head), PhD Student at Research School of Earth Sciences of the Australian National University, Canberra (Alumnus ASPP-APAC 2019).

    \item Ricardo M\'endez Fragoso, Professor at Department of Physics, School of Sciences of the National Autonomous University of Mexico, (Invited to ASPP-Europe 2019).
    
    \item Florencia Noriega, Software Engineering Lecturer at CODE University of Applied Sciences, Berlin (Alumna ASPP-Europe 2016).
  
    \item Tiziano Zito, from the Institut f\"ur Psychologie of the Humboldt Universit\"at zu Berlin (Co-Founder of ASPP-Europe).

    %\item Paula Sanz Le\'on, 
    % \item Erin Mc Kiernan,  Professor in the Department of Physics at the National Autonomous University of Mexico.
    %\item item \note[add]{We should add someone else in Mexico so they can help Rich later on with the local organization (@Rich Could you suggest someone?).}
    %\item
\end{itemize}

\section*{Faculty (Preliminary List)}
\begin{itemize} 
    \item Tiziano Zito, Institut f\"ur Psychologie, Humboldt Universit\"at zu Berlin.    
    \item Rike-Benjamin Schuppner, Institute for Theoretical Biology, Humboldt-Universit\"at zu Berlin, Germany.
    \item St\'efan van der Walt, Researcher, Leader of the Software Working Group, Berkeley Institute for Data Science, University of California at Berkeley, USA.
    \item Josu\'e Mart\'inez Moreno, PhD Student at Research School of Earth Sciences of the Australian National University, Canberra, Australia.
    \item  Florencia Noriega, Lecturer in Software Engineering at CODE University of Applied Sciences, Berlin, Germany.
    \item Olmo Zavala, 
    % \item Lorena Barba (Associate Professor of Mechanical and Aerospace Engineering, School of Engineering and Applied Science, the George Washington University, USA).
    \item Madicken Munk, Postdoctoral Researcher at the  Data Exploration Lab, National Center for Supercomputing Applications, University of Illinois at Urbana-Champaign, USA.
    \item Kathryn D. Huff, Assistant Professor, Department of Nuclear, Plasma, and Radiological Engineering, University of Illinois at Urbana-Champaign, USA.
    \item Fernando P\'erez, Assistant Profesor, Berkeley institute of Data Science, University of California at Berkeley, USA.
    \item Ricardo M\'endez Fragoso, Professor of Physics, Faculty of Sciences of the National Autonomous University of Mexico.
    % \item Carlos Echeverr\'ia Serur
    %\item Luis Garc\'ia Ramos
    \item \add[Josue]{Who else may be interested? Ideally we should invite people from the US, and America Latina}
\end{itemize}

\subsection*{Tutors (Preliminary List)}
\begin{itemize}
\item Carlos Echeverr\'ia Serur.
\item Luis Garc\'ia Ramos.
\item Josu\'e Mart\'inez Moreno.
\item Ricardo M\'endez Fragoso.
\item Florencia Noriega.
\end{itemize}

Note: The budget will fund travel, accommodation and daily expenses 
of lecturers and organizers. Typically, organizers also serve as lecturers. The 
budget may not allow for funding of organizers who do not actively participate 
in the school as lecturers. Additional budget will be use to support 
accommodation or travel expenses of international and local students.

\section*{Vision}
The Advanced Scientific Programming in Python has been a student initiative 
with the goal of better understanding how to make scientific source code 
reproducible, acurate and reusable that many scientists use along their career. 
The school will bring together sceitnrist from all over Latin-America for the
first ASPP in the American continent. During the 7-day intensive course, 
students will explore and learn the fundamentals ideas of scientific code 
development from leading experts in the field. The school will be held in a 
remote location near Mexico City to create a deep, focused learning experience 
away from distractions and daily routines. Informal chats during shared meals, 
recreation time and during coding excercises will provide opportunities to all 
participants to form close professional and personal relationships, fostering 
future collaborations and openness within the next generation of Latin-American 
Scientists.

\section*{Need}
There is a considerable education lack in scientific programming courses. Across
many academic and non-academic institutions, programming has become a key skill
to conduct research. However, in many cases our interaction with codes ranges 
from simply copy-paste code found online to developing and running new modules-packages. 
Whilst many of us are familiar with the basics of programming, less well understand 
the full capabilities and standards used to facilitate openness and sharability of code. 
Additionally, the codes we commonly use do not aliviate the generation of bugs which 
leads to use trial-and-error or rely on advice from others. Therefore, a better 
understanding on better coding practices as well as the underpins on Python, 
will provide the required tools to increase the participants efficiency and code 
robustness while programming.

\section*{Participants}
Participation is free, i.e. no fee is charged, however participates should take care of travel, living, and accommodation expenses by themselves.
We may be able to provide a small travel allowance if resources are sufficient (see ``Funding" below).

\subsection*{Admission Requirements}
These are the requirements for participating in ASPP - Latin America:
\begin{itemize}
    \item Applicants should be research scientists  (Masters, PhD students, Postdocs or Professors) in the natural or social sciences, engineering or mathematics, at an institution in South, Central or North America. Priority will be given to applicants from Latin America and applicants from underrepresented groups.
    \item Applicants are absolutely required to know a high-level programming language used in scientific computing (for example, C, C++, Java, Fortran, MATLAB, etc.) and the basics of Python and version control with git to get the most out of this Summer School.
    \item Applicants should provide proof of their English proficiency by providing a letter of motivation and a short resume (only including current position, and previous studies, no lists of publications) of the applicant written in the English language. Applicants are not required to provide results from a proficiency test. \note[Josue]{Is there something else to add? Shall we ask for a reference letter or something?} \note[Luis]{The application form should require a letter of motivation and a short resume-bio of the participant, this should be enough.}
\end{itemize}

\section*{Lectures }%of the Summer School}
These lectures will build from feedback obtained during the previous 
ASPP-Europe and ASPP-Asia Pacific schools (topics include some of the 
potential speakers).

\begin{itemize}
    \item Version control (GitHub) - 
    \item Tidy data analysis and visualization - Rich \note[Josue]{Perhaps 
    change the structure and remove Tidy data?s}
    \item Testing and debugging scientific code - Luis
    \item Advanced NumPy - Carlos
    \item Organizing, documenting, distributing  scientific code and 
    contimuous integration - 
    \item Advanced scientific Python: context managers and generators - 
    \item Writing parallel applications in Python - 
    \item Profiling and speeding up scientific code with Cython and numba - 
\end{itemize}

\subsection*{Schedule}

\begin{center}
\begin{tabularx}{\textwidth}{|X|X|X|X|}
\hline
\rowcolor{Aquamarine}
\multicolumn{4}{|c|}{Monday XX July 2020}\\
\hline
\rowcolor[gray]{.7}
Time & Topic & Speaker & Tutors \\
\hline
9:00-9:30 & Introduction &  &   \\
\hline
9:30-11:00 & Git \& Github &  &  \\
\hline
\rowcolor[gray]{.9}
11:00-11:30 & \multicolumn{3}{c|}{Coffee Break} \\
\hline
11:30-13:00 & Git \& Github &  &  \\
\hline
\rowcolor[gray]{.9}
13:00-14:00 & \multicolumn{3}{c|}{Lunch Break} \\
\hline
14:00-15:00 & Git \& Github &  &  \\
\hline
15:00-16:00 & Data visualization & Ricardo M\'endez &  \\
\hline
\rowcolor[gray]{.9}
16:00-16:30 & \multicolumn{3}{c|}{Coffee Break} \\
\hline
16:30-18:00 & Data visualization & Ricardo M\'endez &  \\
\hline
\end{tabularx}
\end{center}

\begin{center}
\begin{tabularx}{\textwidth}{|X|X|X|X|}
\hline
\rowcolor{Aquamarine}
\multicolumn{4}{|c|}{Tuesday XX July 2020}\\
\hline
\rowcolor[gray]{.7}
Time & Topic & Speaker & Tutors \\
\hline
9:00-9:30 & Testing \& debugging &  &   \\
\hline
9:30-11:00 & Testing \& debugging &  &  \\
\hline
\rowcolor[gray]{.9}
11:00-11:30 & \multicolumn{3}{c|}{Coffee Break} \\
\hline
11:30-13:00 & Testing \& debugging &  &  \\
\hline
\rowcolor[gray]{.9}
13:00-14:00 & \multicolumn{3}{c|}{Lunch Break} \\
\hline
14:00-15:00 & Advanced Numpy &  &  \\
\hline
15:00-16:00 & Advanced Numpy  &  &  \\
\hline
\rowcolor[gray]{.9}
16:00-16:30 & \multicolumn{3}{c|}{Coffee Break} \\
\hline
16:30-18:00 &  &  &  \\
\hline
\end{tabularx}
\end{center}
\note[Josue]{I would like to add some Scipy, what do you think?}

\begin{center}
\begin{tabularx}{\textwidth}{|X|X|X|X|}
\hline
\rowcolor{Aquamarine}
\multicolumn{4}{|c|}{Wendsday XX July 2020}\\
\hline
\rowcolor[gray]{.7}
Time & Topic & Speaker & Tutors \\
\hline
9:00-11:00 & Organizing, documenting, distributing \& CI &  &  \\
\hline
\rowcolor[gray]{.9}
11:00-11:30 & \multicolumn{3}{c|}{Coffee Break} \\
\hline
11:30-13:00 &  &  &  \\
\hline
\rowcolor[gray]{.9}
13:00-XX:XX & \multicolumn{3}{c|}{Social activity} \\
\hline
\end{tabularx}
\end{center}
\note[Josue]{Do we want a social activity?}

\begin{center}
\begin{tabularx}{\textwidth}{|X|X|X|X|}
\hline
\rowcolor{Aquamarine}
\multicolumn{4}{|c|}{Thursday XX July 2020}\\
\hline
\rowcolor[gray]{.7}
Time & Topic & Speaker & Tutors \\
\hline
9:00-10:30 & Parallel Python &  &   \\
\hline
\rowcolor[gray]{.9}
10:30-11:00 & \multicolumn{3}{c|}{Coffee Break} \\
\hline
11:30-13:00 & Parallel Python - Pycuda &  &  \\
\hline
\rowcolor[gray]{.9}
13:00-14:00 & \multicolumn{3}{c|}{Lunch Break} \\
\hline
14:00-14:30 & Project Introduction &  &  \\
\hline
14:30-16:00 & Programming Project &  &  \\
\hline
\rowcolor[gray]{.9}
16:00-16:30 & \multicolumn{3}{c|}{Coffee Break} \\
\hline
16:30-18:00 & Programming Project &  &  \\
\hline
\end{tabularx}
\end{center}

\begin{center}
\begin{tabularx}{\textwidth}{|X|X|X|X|}
\hline
\rowcolor{Aquamarine}
\multicolumn{4}{|c|}{Friday XX July 2020}\\
\hline
\rowcolor[gray]{.7}
Time & Topic & Speaker & Tutors \\
\hline
9:00-10:30 & Profiling, Cython \& numba &  &   \\
\hline
\rowcolor[gray]{.9}
10:30-11:00 & \multicolumn{3}{c|}{Coffee Break} \\
\hline
11:30-13:00 & Profiling, Cython \& numba  &  &  \\
\hline
\rowcolor[gray]{.9}
13:00-14:00 & \multicolumn{3}{c|}{Lunch Break} \\
\hline
14:00-16:00 & Programming Project &  &  \\
\hline
\rowcolor[gray]{.9}
16:00-16:30 & \multicolumn{3}{c|}{Coffee Break} \\
\hline
16:30-18:00 & Programming Project &  &  \\
\hline
\end{tabularx}
\end{center}

\begin{center}
\begin{tabularx}{\textwidth}{|X|X|X|X|}
\hline
\rowcolor{Aquamarine}
\multicolumn{4}{|c|}{Saturday XX July 2020}\\
\hline
\rowcolor[gray]{.7}
Time & Topic & Speaker & Tutors \\
\hline
9:00-10:30 & Programming Project &  &   \\
\hline
\rowcolor[gray]{.9}
10:30-11:00 & \multicolumn{3}{c|}{Coffee Break} \\
\hline
11:30-13:00 & Programming Project &  &  \\
\hline
\rowcolor[gray]{.9}
13:00-14:00 & \multicolumn{3}{c|}{Lunch Break} \\
\hline
14:00-16:00 & Programming Project &  &  \\
\hline
\rowcolor[gray]{.9}
16:00 & \multicolumn{3}{c|}{Repository Freeze} \\
\rowcolor[gray]{.9}
16:00-16:30 & \multicolumn{3}{c|}{Coffee Break} \\
\hline
16:30-16:45 & How to contribute to ASPP &  &  \\
\hline
\rowcolor[gray]{.9}
16:45-17:00 & \multicolumn{3}{c|}{Programming project - Learning report}\\
\hline
\rowcolor[gray]{.9}
17:00-18:00 & \multicolumn{3}{c|}{Friendly Tournament} \\
\hline
\rowcolor[gray]{.9}
18:00-XX:XX & \multicolumn{3}{c|}{Final Social Event} \\
\hline
\end{tabularx}
\end{center}

\note[Josue]{I've removed 1 hour from the lunch break so people can go earlier home, do you think we should use that hour or shouldn't?}
\note[Luis and Carlos]{We think they should have two hours of lunch break (to be discussed)}


%\subsection*{Additional materials}
%\note[Carlos]{what should go here?}

\section*{Estimated Costs}
The budget will fund travel, accommodation and daily expenses of lecturers and organizers. Organizers can also serve as lecturers or tutors since the budget may not allow for funding of organizers who do not actively participate in the school as lecturers. Additional resources will be used to support accommodation or travel expenses of international and local students.

\subsection*{Estimated costs per Faculty Member (should be covered by the organizers)}
\begin{itemize}
    \item Flight (from Europe or US): Approx. 800USD 
    \item Accommodation in CDMX: Approx. 400USD
    \item Local Transport: Approx. 100USD
    \item Food for the entire duration of the conference: Approx. 200USD
\end{itemize}

\subsection*{Estimated costs of Organization (sholuld be covered by the organizers)}
\begin{itemize}
  \item Computer Rental (40 computers): Approx. ???USD
  \item Venue Rental: Approx. ???USD
  \item Social Activity (Trip to Teotihuac\'an): Approx. ???USD
  \item Airport-Venue Transport: Approx. ???USD
\end{itemize}

\subsection*{Estimated average costs per Participant (should be covered by the participants)}
\begin{itemize}
  \item Flight (From a Latin American Country): Approx. 500USD
  \item Accommodation in CDMX: Approx. 300USD
  \item Local Transport: Approx. 50USD
  \item Food for the entire duration of the conference: Approx. 150USD
\end{itemize}

\subsection*{Tentative Partner Organizations (Sponsors/Funding)}
\begin{itemize}
    \item Secretar\'ia de Ciencia, Tecnolog\'ia e Innovaci\'on de la CDMX.
    \item Faculty of Sciences (FCIENCIAS-UNAM).
    \item Instituto de Investigaciones en Matem\'aticas Aplicadas (IIMAS-UNAM)
    \item Consejo Nacional de Ciencia y Tecnolog\'ia (CONACYT)
    \item Instituto Latinoamericano de la Comunicaci\'on Educativa (ILCE)
    \item Instituto Polit\'ecnico Nacional (IPN)
\end{itemize}

\section*{Policies}

\subsection*{Inclusiveness}
 We are striving hard to obtain a group of participants which is international and gender-balanced.

\subsection*{Behavior}
See \url{https://numfocus.org/code-of-conduct}{Code of conduct}

\subsection*{}


\end{document}
